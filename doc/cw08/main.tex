\documentclass{beamer}

% Required packages 
\usepackage{fontspec}
\usepackage{fontawesome}
\usepackage{hyperref}
%\usepackage[ngerman]{babel}
\usetheme{metropolis}           

\newcommand{\cw}{50}

% Header
\title{AMSE - Kalendarwoche \cw{}}
\date{11. Dezember 2019}
\author{Benjamin Fischer}
\institute{benjamin.f.fischer@fau.de}

% Document
\begin{document}
  \maketitle

  % Structure (always the same)
  \begin{frame}
    \begin{itemize}
      \item Zusammenfassung Kalendarwoche \cw{}
      \begin{itemize}
        \item Erreichte Ziele
        \item Aufgetretene Probleme
      \end{itemize}
      \item Ausblick nächste Woche
    \end{itemize}
  \end{frame}


  \begin{frame}
    \frametitle{Zusammenfassung Kalendarwoche \cw{} (Frontend)}
    \begin{itemize}
      \item Arbeit an Frontend begonnen (nun auch mit Docker)
      \begin{itemize}
        \item Zugriff nur von localhost (ohne \textit{iptables})
      \end{itemize}
      \item Minimales Setup: App mit TabNavigator und 4 \textit{Screens}
      \begin{itemize}
        \item \texttt{Home}: Suche nach Abgeordneten
        \item \texttt{Profile}: Übersicht des Profils
        \item \texttt{Personal}: Darstellung von Nebenjobs, Abstimmungen, etc.
        \item \texttt{Speech}: Anzeige einer zuvor ausgewählten Rede
      \end{itemize}
      \item Verwendung von \texttt{redux} um internen Zustand zu verwalten
      \item Linter eingebunden (eslint)
      \item \texttt{DEMO}
    \end{itemize}
  \end{frame}

  \begin{frame}
    \frametitle{Ausblick nächste Woche}
    \begin{itemize}
      \item Funktionalitäten der anderen Tabs implementieren
      \begin{itemize}
        \item Darstellung in einzelne Komponenten verlagern
        \item Layout (v.a \textit{Buttons} und \textit{Dropdowns})
      \end{itemize}
    \end{itemize}
  \end{frame}

\end{document}
