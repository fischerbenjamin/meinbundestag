\documentclass{beamer}

% Required packages 
\usepackage{fontspec}
\usepackage{fontawesome}
\usepackage{hyperref}
%\usepackage[ngerman]{babel}
\usetheme{metropolis}           

\newcommand{\cw}{45}

% Header
\title{AMSE - Kalendarwoche \cw{}}
\date{06. November 2019}
\author{Benjamin Fischer}
\institute{benjamin.f.fischer@fau.de}

% Document
\begin{document}
  \maketitle

  % Structure (always the same)
  \begin{frame}
    \begin{itemize}
      \item Zusammenfassung Kalendarwoche \cw{}
      \begin{itemize}
        \item Erreichte Ziele
        \item Aufgetretene Probleme
      \end{itemize}
      \item Ausblick nächste Woche
    \end{itemize}
  \end{frame}

  % TODO: Insert the achieved goals here and mention the problems that occured.
  \begin{frame}
    \frametitle{Zusammenfassung Kalendarwoche \cw{}}
    \begin{itemize}
      \item Verarbeitung der Protokolldateien
      \begin{itemize}
        \item Bei DTD fehlte im Attribut \textit{herstellung} "KG"
        \item Keyword-Analyse nicht so \textit{einfach} wie erhofft
        \item Aktuelle Analyse mit \textit{textblob} (siehe Beispiel)
      \end{itemize}
      \item Struktur des Source-Codes
      \begin{itemize}
        \item Einteilung in \textit{src}, \textit{data} und \textit{test}
        \item Kleinere Pfad-Probleme
        \item Aktuell: gemeinsamer Einstiegspunkt \textit{main.py}
      \end{itemize}
      \item Keine Änderungen in Frontend
    \end{itemize}
  \end{frame}

  % TODO: Insert the goals for the new week.
  \begin{frame}
    \frametitle{Ausblick nächste Woche}
    \begin{itemize}
      \item \textit{Unittests} einbauen
      \item \textit{Docker compose} um Services miteinander zu verbinden
      \begin{itemize}
        \item Datenbank (mongodb)
        \item Flask Applikation (stellt REST API zur Verfügung)
      \end{itemize}
      \item Wünschenswert: Backend läuft schon als Prototyp
    \end{itemize}
  \end{frame}

\end{document}
