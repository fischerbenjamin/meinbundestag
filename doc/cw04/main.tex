\documentclass{beamer}

% Required packages 
\usepackage{fontspec}
\usepackage{fontawesome}
\usepackage{hyperref}
%\usepackage[ngerman]{babel}
\usetheme{metropolis}           

\newcommand{\cw}{46}

% Header
\title{AMSE - Kalendarwoche \cw{}}
\date{13. November 2019}
\author{Benjamin Fischer}
\institute{benjamin.f.fischer@fau.de}

% Document
\begin{document}
  \maketitle

  % Structure (always the same)
  \begin{frame}
    \begin{itemize}
      \item Zusammenfassung Kalendarwoche \cw{}
      \begin{itemize}
        \item Erreichte Ziele
        \item Aufgetretene Probleme
      \end{itemize}
      \item Ausblick nächste Woche
    \end{itemize}
  \end{frame}

  \begin{frame}
    \frametitle{Zusammenfassung Kalendarwoche \cw{}}
    \begin{itemize}
      \item \texttt{docker-compose} für Backend erstellt
      \item Zugriff auf verarbeitete Reden über REST-API
      \begin{itemize}
        \item Verarbeitung und Einfügen der Protokolle im Hintergrund
      \end{itemize}
      \item Aktuell: Flask-App und \textit{Updater} in eigenem Thread
      \item ODS versucht zu konfigurieren, leider erfolglos
    \end{itemize}
  \end{frame}

  \begin{frame}
    \frametitle{Ausblick nächste Woche}
    \begin{itemize}
      \item ODS mit neuem Image testen
      \item Alternative: Route in REST-API einbauen
      \item Verlagerung der einzelnen Funktionalitäten in Threads
      \item Kommunikation dann über Datenbank
      \item Verwaltung der DB über REST-API ermöglichen 
              (\textit{clear}, \textit{update}, etc.)
    \end{itemize}
  \end{frame}

\end{document}
