\documentclass{beamer}

% Required packages 
\usepackage{fontspec}
\usepackage{fontawesome}
\usepackage{hyperref}
%\usepackage[ngerman]{babel}
\usetheme{metropolis}           

\newcommand{\cw}{51}

% Header
\title{AMSE - Kalendarwoche \cw{}}
\date{18. Dezember 2019}
\author{Benjamin Fischer}
\institute{benjamin.f.fischer@fau.de}

% Document
\begin{document}
  \maketitle

  % Structure (always the same)
  \begin{frame}
    \begin{itemize}
      \item Zusammenfassung Kalendarwoche \cw{}
      \begin{itemize}
        \item Erreichte Ziele
        \item Aufgetretene Probleme
      \end{itemize}
      \item Ausblick nächste Woche
    \end{itemize}
  \end{frame}


  \begin{frame}
    \frametitle{Zusammenfassung Kalendarwoche \cw{} (Frontend)}
    \centering
    \begin{LARGE}
      DEMO
    \end{LARGE}
  \end{frame}

  \begin{frame}
    \frametitle{Ausblick Weihnachtsferien}
    \begin{itemize}
      \item Ziel: Implementierung abschließen
      \item Im Frontend:
      \begin{itemize}
        \item Lade-Indikatoren und Willkommens-Bildschirm
        \item Such-Funktion selbst implementieren
        \item Refaktorisieren und Dokumentieren
        \item Bugfixes
      \end{itemize}
      \item Im Backend
      \begin{itemize}
        \item ODS wieder anbinden
        \item Keine \textit{Reden} aus Anhängen und nur mit Mindestlänge
        \begin{itemize}
          \item Datenbank updaten
        \end{itemize}
      \end{itemize}
    \end{itemize}
  \end{frame}

  \begin{frame}
    \frametitle{Ausblick Januar}
    \begin{itemize}
      \item Dokumentation des Projekts
      \item Vorbereiten der Abschlusspräsentation
      \item kleinere Anpassungen \& Bugfixes
    \end{itemize}
  \end{frame}

\end{document}
